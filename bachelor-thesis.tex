\documentclass{article}
\usepackage[english]{babel}
\usepackage[a4paper,top=2cm,bottom=2cm,left=3cm,right=3cm,marginparwidth=1.75cm]{geometry}

% Useful packages
\usepackage{amsmath}
\usepackage{graphicx}
\usepackage[colorlinks=true, allcolors=blue]{hyperref}

\title{Exploration of the alignment problem in machine learning with the use of simulated agents using a critical dissection of the prevailing anthropocentric approach}
\author{Felix Willen}

\begin{document}
\maketitle

\begin{abstract}
It is easy to teach a machine basic things. It is easy for the machine to do something, if it knows what it is seeing or sensing. The alignment problem starts there, where the machine comes into unknown territory. We, as humans, need to teach machines our values, in order to guarantee an important alignment with our vision.
\end{abstract}

\section{Introduction}

Historical context introduction about computers and how we ended up with neural networks (starting in 1957 with the perceptron), how we copy and learn from the nature

\section{The computer is the world}
 Anthropocentric approach on Artificial Intelligence and the opposition: the computer is the world

\subsection{Human cantered approach to machine learning}

Simply use the section and subsection commands, as in this example document! With Overleaf, all the formatting and numbering is handled automatically according to the template you've chosen. If you're using Rich Text mode, you can also create new section and subsections via the buttons in the editor toolbar. In order to operate on this scale, there have been some changes

\section{Moving fast and breaking things}

The responsibility that comes with new technology and their usage, Silicon Valley mantra: moving fast and breaking things

\section{Doing what they said, not what they wanted}

Exploration of Machine Learning Alignment with Unity and ml-agents

\section{Conclusion}

Conclusion about limitations and where AI works best

\newpage
\bibliographystyle{cell}
\bibliography{bachelor-thesis.bib}

\end{document}